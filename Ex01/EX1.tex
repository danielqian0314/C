% !TEX TS-program = pdflatex
% !TEX encoding = UTF-8 Unicode

% This is a simple template for a LaTeX document using the "article" class.
% See "book", "report", "letter" for other types of document.

\documentclass[11pt]{article} % use larger type; default would be 10pt

\usepackage[utf8]{inputenc} % set input encoding (not needed with XeLaTeX)

%%% Examples of Article customizations
% These packages are optional, depending whether you want the features they provide.
% See the LaTeX Companion or other references for full information.

%%% PAGE DIMENSIONS
\usepackage{geometry} % to change the page dimensions
\geometry{a4paper} % or letterpaper (US) or a5paper or....
% \geometry{margin=2in} % for example, change the margins to 2 inches all round
% \geometry{landscape} % set up the page for landscape
%   read geometry.pdf for detailed page layout information

\usepackage{graphicx} % support the \includegraphics command and options

% \usepackage[parfill]{parskip} % Activate to begin paragraphs with an empty line rather than an indent

%%% PACKAGES
\usepackage{booktabs} % for much better looking tables
\usepackage{array} % for better arrays (eg matrices) in maths
\usepackage{paralist} % very flexible & customisable lists (eg. enumerate/itemize, etc.)
\usepackage{verbatim} % adds environment for commenting out blocks of text & for better verbatim
\usepackage{subfig} % make it possible to include more than one captioned figure/table in a single float
\usepackage[fleqn]{amsmath}
\usepackage{amsfonts}
% These packages are all incorporated in the memoir class to one degree or another...

%%% HEADERS & FOOTERS
\usepackage{fancyhdr} % This should be set AFTER setting up the page geometry
\pagestyle{fancy} % options: empty , plain , fancy
\renewcommand{\headrulewidth}{0pt} % customise the layout...
\lhead{}\chead{}\rhead{}
\lfoot{}\cfoot{\thepage}\rfoot{}

%%% SECTION TITLE APPEARANCE
\usepackage{sectsty}
\allsectionsfont{\sffamily\mdseries\upshape} % (See the fntguide.pdf for font help)
% (This matches ConTeXt defaults)

%%% ToC (table of contents) APPEARANCE
\usepackage[nottoc,notlof,notlot]{tocbibind} % Put the bibliography in the ToC
\usepackage[titles,subfigure]{tocloft} % Alter the style of the Table of Contents
\renewcommand{\cftsecfont}{\rmfamily\mdseries\upshape}
\renewcommand{\cftsecpagefont}{\rmfamily\mdseries\upshape} % No bold!

%%% END Article customizations

%%% The "real" document content comes below...

\title{Assignment H1 - Solution}
\author{Gong,Zhaowen	3305359 \\
        He,Jiayu	3300561 \\
        Qian,Dongcheng	3302747}

%\date{} % Activate to display a given date or no date (if empty),
         % otherwise the current date is printed 

\begin{document}
\maketitle

\section{Problem 1 (Discrete Convolution)}

Solution:\\
\begin{flalign}
\begin{split}
(f*f)_i  & = \sum_{k=-\infty}^{+\infty}f_{i-k} \cdot f_k  \\
         & = f_{i-0} \cdot f_0 + f_{i-1} \cdot f_1  \\
		& = \frac 12 f_i + \frac 12 f_{i-1}\\ 
\\
(f*(f*f)_i)_i  & = \sum_{k=-\infty}^{+\infty}f_{i-k} \cdot (\frac 12 f_k + \frac 12 f_{k-1})  \\
         & = f_{i-0} \cdot \frac 12 f_0 + f_{i-1} \cdot (\frac 12 f_1 + \frac 12 f_0) + \frac 12 f_{i-2} \cdot \frac 12 f_1  \\
		& = \frac 14 f_i + \frac 12 f_{i-1} + \frac 14 f_{i-2}\\ 
\nonumber
\end{split}&
\end{flalign}
\section{Problem 2 (Properties of the Convolution )}

Solution:\\
\\
a) Linearity:\\
\begin{flalign}
\begin{split}
(\alpha \cdot f + \beta \cdot g)*\omega &= \sum_{k=-\infty}^{+\infty}(\alpha \cdot f_{i-k}+\beta \cdot g_{i-k})\cdot \omega_k\\
&= \alpha \cdot \sum_{k=-\infty}^{+\infty} f_{i-k} \cdot \omega_k + \beta \cdot \sum_{k=-\infty}^{+\infty} g_{i-k} \cdot \omega_k\\
&= \alpha \cdot (f*\omega) + \beta \cdot (g*\omega) \qquad for \; all \; \alpha,\beta \in \mathbb{R}\\
\nonumber
\end{split}&
\end{flalign}
b) Commutativity:\\
\begin{flalign}
\begin{split}
f*\omega &= \sum_{k=-\infty}^{+\infty}f_{i-k} \cdot \omega_k = \sum_{k=-\infty}^{+\infty} \omega_k \cdot f_{i-k}\\
\nonumber
\end{split}&
\end{flalign}
define $\quad n:=i-k\qquad n\in \mathbb{Z}$
\begin{flalign}
\begin{split}
\sum_{k=-\infty}^{+\infty} \omega_k \cdot f_{i-k}&= \sum_{i-n=-\infty}^{+\infty} \omega_{i-n} \cdot f_n\\
&= \sum_{n=i+\infty}^{i-\infty} \omega_{i-n} \cdot f_n\\
&= \sum_{n=-\infty}^{+\infty} \omega_{i-n} \cdot f_n\\ &= \omega * f\\
\nonumber
\end{split}&
\end{flalign}
c) Identity:
\begin{flalign}
\begin{split}
f*e &= \sum_{k=-\infty}^{+\infty}f_{i-k}\cdot e_k=f_i\\
e_i &= \left\{ \begin{array}{ll}
1 & \textrm{$(i=0)$}\\
0 & \textrm{(else)}
\end{array} \right.
\nonumber
\end{split}&
\end{flalign}








\end{document}
